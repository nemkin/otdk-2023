\chapter{Motivation}

Bioinformatics is an interdisciplinary field between computer science and biology. Its main goal is to answer biological questions by transforming them into computational problems and providing efficient algorithmic solutions to them. Researching bioinformatics has a significant impact on our everyday lives since discoveries in this area can help us solve many of today’s major global problems, for example, aiding the creation of more effective medical treatments, advancing our understanding of genetic diseases, developing resistant crops to tackle a global food crisis or inventing novel technologies to reduce and revert environmental pollution. Unfortunately, many of the applicable problems in bioinformatics turn out to be computationally hard problems. Despite all the research effort, no sufficiently fast, deterministic solutions were found to these problems with the current limitations of our hardware. In recent years, the field of quantum informatics has experienced increased attention from governmental entities and global corporations, who invested significantly into designing new types of hardware based on quantum physical phenomena and developing complementary software capable of solving previously challenging problems. While quantum computing research is still in its early stages and the limits of quantum hardware are unknown, the availability of this different computational model has allowed new forms of algorithmic design to emerge, with promising theoretical results. In my dissertation, I introduce the mathematical framework of quantum computation, particularly quantum walks and search algorithms. I investigate the implementation details of quantum walks and showcase them via a simulator software which can aid understanding and visualizing the unfamiliar behaviour of these algorithms. I present a general overview of computational bioinformatics problems, particularly protein folding, detailing the different models of protein behaviour, and analyse the classical and quantum possibilities for solutions. 
