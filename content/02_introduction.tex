\chapter{Introduction}

Quantum Computing is a rapidly growing technology in the 21th century. 

One of the famous Millennium Prize Problems  Clay Mathematics Institute,  is the P versus NP problem, which essentially asks whether many of the currently know algorithmic problems can be solved in a fast time? This problem has eluded computer scientists for a century. But it only exists in classical Turing-machines, quantum has an entirely different computational model we can cirqumvent it by using the Quantum Turing-machine.

Richard Feynman first introduced the idea in a 1982 article, explaining that currently existing classical computers are ill-equipped to deal with the calculations required to run quantum physical simulations. But if we had a computer that was run by the same quantum physical phenomena it was meant to work on, it could be able give the performance necessary to run these computations.

It turns out that when we build a computer based off of quanum physical phenomena, an entirely new model is necessary to describe its function. Following the work of many computer scientists (Benioff, Deutsch, Bernstein and Vazirani), the computational model for Quantum computers was mathematically defined in the late 1980s: the Quantum Turing machine.

Shor's quantum prime factorization algorithm in 1990 first drew attention to the potential of quantum computers. If someone could build such a machine in practice, it could quickly crack the RSA encryption that is widely used today.

In 2019, Google's 54 quantum-bit Sycamore processor successfully beat today's fastest supercomputer to perform a specific computation, achieving quantum supremacy.

Nowadays, quantum is becoming a hot topic and more and more funding is being allocated to quantum research in Hungary. At the BME, the National Laboratory for Quantum Informatics is working on the development of an internet based on quantum encryption and many departments, including SZIT, are involved in the project.
