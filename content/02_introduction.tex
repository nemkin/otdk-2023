\chapter{Introduction}


Why are we interested in quantum computers?

Originally the idea of a quantum computer was suggested by Richard Feynman in a 1982 article, where he explains that currently existing classical computers are ill-equipped to deal with the complexity of the calculations required to simulate quantum physics. His suggestion was to replace the current hardware standard with one, which works based on quantum physical phenomena, thus giving it the capability to simulate the very thing it is based on.

The current computational model we use for classical computers is called the Turing-machine. These new types of computers are so different from classical ones that they run based on a completely different set of rules. Following the work of many computer scientists (Benioff, Deutsch, Bernstein and Vazirani), the computational model for Quantum computers was mathematically defined in the late 1980s: the Quantum Turing machine.

In the classical world, on the Turing machine, mathematicians and computer scientists have been struggling with the same problem for more than a hundred years: a solution to a set of computational problems that could unlock many real-world use cases in many industries [TODO: list examples]. These are called the NP-complete. This problem has eluded computer scientists for a century. This problem is so big that it is one of the famous Millennium Prize Problems set by the Clay Mathematics Institute, the P versus NP problem.

In the quantum world, the rules of the Quantum Turing machine are different [TODO Scott Aaronson article about what we can do with Quantum computers]. This gives us hope that we could use Quantum Computers to solve the problems fast we are currently struggling in the classical world with.

In recent years, there has been an increasing focus on quantum informatics. Influential global companies such as IBM, Google, Microsoft, and Amazon have invested significant amounts into studying and developing hardware and software for this sector, while the European Union and Hungary have launched several programs to accelerate quantum research.
